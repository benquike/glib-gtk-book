\chapter{Semi-Object-Oriented Programming in C}

It was explained in the previous chapter that the GLib core library uses a semi-object-oriented coding style. This section explains what it means, and how to write your own code with this coding style.

One of the main ideas of OOP is to \emph{keep related data and behavior in one place}\footnote{This is one of the guidelines discussed in \emph{Object-Oriented Design Heuristics} \cite{oop-book}.}. In C, the data is stored in a \lstinline{struct}, and the behavior is implemented with functions. To keep those in one place, we put them in the same *.c file, with the public functions present in the corresponding *.h file (the header).

Another important idea of OOP is to \emph{hide all data within its class}. In C, it means that the \lstinline{struct} full declaration should be present only in the *.c file, while the header contains only a \lstinline{typedef}. How the data is stored within the class and which data structures are used should remain an implementation detail. A user of the class should not be aware of the implementation details, it should instead rely only on the external interface, that is, what is present in the header and the public documentation. That way, the implementation of the class can change without affecting the users of the class, as long as the API doesn't change.

\section{Header Example}

The Listing~\ref{oop-semi-spell-checker-h} p.~\pageref{oop-semi-spell-checker-h} shows an example of a header providing a simple spell checker. This is a fictitious code; if you need a spell checker in your GTK+ application you would nowadays probably use the gspell library\footnote{\url{https://wiki.gnome.org/Projects/gspell}}.

\lstinputlisting[float, caption={myapp-spell-checker.h}, label=oop-semi-spell-checker-h]{code/myapp-spell-checker.h}

\subsection{Project Namespace}
The first thing to note is the use of the namespace ``Myapp''. Each symbol in the header is prefixed by the project namespace.

It is a good practice to choose a namespace for your code, to avoid symbol conflicts at link time. It is especially important to have a namespace for a library, but it is also better to have one for an application. Of course the namespace needs to be unique for each codebase; for instance you must \emph{not} re-use the ``G'' or ``Gtk'' namespaces for your application or library!

\subsection{Class Namespace}
Additionally, there is a second namespace with the class name, here ``SpellChecker''. If you follow that convention consistently, the name of a symbol is more predictable, which makes the API easier to work with.

\subsection{Lowercase, Uppercase or CamelCase?}
Depending on the symbol type, its name is either in uppercase, lowercase or CamelCase. The convention in the GLib world is:
\begin{itemize}
  \item Uppercase letters for a constant, either for a \lstinline{#define} or an \lstinline{enum} value;
  \item CamelCase for a \lstinline{struct} or \lstinline{enum} type;
  \item Lowercase for functions, variables, \lstinline{struct} fields, …
\end{itemize}

\subsection{Include Guard}
The header contains the typical include guard:

\begin{lstlisting}
#ifndef MYAPP_SPELL_CHECKER_H
#define MYAPP_SPELL_CHECKER_H

/* ... */

#endif /* MYAPP_SPELL_CHECKER_H */
\end{lstlisting}

It protects the header from being included several times in the same *.c file.

\subsection{C++ Support}
The \lstinline{G_BEGIN_DECLS}/\lstinline{G_END_DECLS} pair permits the header to be included from C++ code. It is more important for a library, but it is also a good practice to add those macros in application code too, even if the application doesn't use C++. That way an application class could be moved to a library easily (it can be hard to notice that the \lstinline{G_BEGIN_DECLS} and \lstinline{G_END_DECLS} macros are missing). And if the desire arises, the application could be ported to C++ incrementally.

\subsection{\#include}
\label{oop-semi-include-in-header}
There are several ways to organize the \lstinline{#include}'s in a C codebase. The convention in GLib/GTK+ is that including a header in a *.c file should not require including another header beforehand.

\path{myapp-spell-checker.h} contains the following \lstinline{#include}:
\begin{lstlisting}
#include <glib.h>
\end{lstlisting}

Because \path{glib.h} is needed for the \lstinline{G_BEGIN_DECLS} and \lstinline{G_END_DECLS} macros, for the GLib basic type definitions (\lstinline{gchar}, \lstinline{gboolean}, etc) and \lstinline{GSList}.

If the \lstinline{#include} in \path{myapp-spell-checker.h} is removed, each *.c file that includes \path{myapp-spell-checker.h} would also need to include \path{glib.h} beforehand, otherwise the compiler would not be able to compile that *.c file. But we don't want to add such requirement for the users of the class.

\subsection{Type Definition}
The \lstinline{MyappSpellChecker} type is defined as follows:

\begin{lstlisting}
typedef struct _MyappSpellChecker MyappSpellChecker;
\end{lstlisting}

The \lstinline{struct _MyappSpellChecker} will be declared in the \path{myapp-spell-checker.c} file. When you use \lstinline{MyappSpellChecker} in another file, you should not need to know what the \lstinline{struct} contains, you should use the public functions of the class instead. The exception to that OOP best practice is when calling a function would be a performance problem, for example for low-level data structures used for computer graphics (coordinates, rectangles, …). But remember: \emph{premature optimization is the root of all evil} (Donald~Knuth).

\subsection{Object Constructor}
\lstinline{myapp_spell_checker_new()} is the constructor of the class. It takes a language code parameter ---~for example \lstinline{"en_US"}~--- and returns an \emph{instance} of the class, also called an \emph{object}. What the function does is simply to allocate dynamically the \lstinline{struct} and return the pointer. That return value is then used as the first parameter of the remaining functions of the class. In some languages like Python, that first parameter is called the \emph{self} parameter, since it references ``itself'', i.e. its own class. Other object-oriented languages such as Java and C++ have the \emph{this} keyword to access the object data.

Note that \lstinline{myapp_spell_checker_new()} can be called several times to create different objects. Each object holds its own data. That's the fundamental difference with global variables: if the data is stored in global variables, there can be at most one instance of it\footnote{Unless the global variable stores the ``objects'' in an aggregated data structure like an array, a linked list or a hash table, with an identifier as the ``\emph{self}'' parameter (e.g. an integer or a string) to access a specific element. But this is really ugly code, please don't do that!}.

%TODO explain new() (allocation on the heap) vs init() (allocation on the stack) and give an example with init(). But the example requires also the *.c code, so it's maybe better to add the explanation in a later subsection.

\subsection{Object Destructor}
\lstinline{myapp_spell_checker_free()} is the destructor of the class. It destroys an object by freeing its memory and releasing other allocated resources.

\subsection{Other Public Functions}
The \lstinline{myapp_spell_checker_check_word()} and \lstinline{myapp_spell_checker_get_suggestions()} functions are the available features of the class. It checks whether a word is correctly spelled, and get a list of suggestions to fix a misspelled word.

The \lstinline{word_length} parameter type is \lstinline{gssize}, which is a GLib integer type that can hold ---~for instance~--- the result of \lstinline{strlen()}, and can also hold a negative value since ---~contrary to \lstinline{gsize}~--- \lstinline{gssize} is a \emph{signed} integer type. The \lstinline{word_length} parameter can be \lstinline{-1} if the string is nul-terminated, that is, if the string is terminated by the special character \lstinline{'\0'}. The purpose of the \lstinline{word_length} parameter is to be able to pass a pointer to a word that belongs to a larger string, without the need to call for example \lstinline{g_strndup()}.

\section{The Corresponding *.c File}

Let's now look at the \path{myapp-spell-checker.c} file:

\vspace{0.7cm}
\lstinputlisting[caption={myapp-spell-checker.c}, label=oop-semi-spell-checker-c]{code/myapp-spell-checker.c}

\subsection{Order of \#include's}
At the top of the file, there is the usual list of \lstinline{#include}'s. A small but noteworthy detail is that the include order was not chosen at random. In a certain *.c file, it is better to include first its corresponding *.h file, and then the other headers\footnote{Except if you have a \path{config.h} file, in that case you should \emph{first} include \path{config.h}, \emph{then} the corresponding *.h, and then the other headers.}. By including first \path{myapp-spell-checker.h}, if an \lstinline{#include} is missing in \path{myapp-spell-checker.h}, the compiler will report an error. As explained on p.~\pageref{oop-semi-include-in-header}, a header should always have the minimum required \lstinline{#include}'s for that header to be included in turn.

Also, since \path{glib.h} is already included in \path{myapp-spell-checker.h}, there is no need to include it a second time in \path{myapp-spell-checker.c}.

\subsection{GTK-Doc Comments}
The public API is documented with GTK-Doc comments. A GTK-Doc comment begins with \lstinline{/**}, with the name of the symbol to document on the next line. When we refer to a symbol, there is a special syntax to use depending on the symbol type:
\begin{itemize}
  \item A function parameter is prefixed by \lstinline{@}.
  \item The \emph{name} of a \lstinline{struct} or \lstinline{enum} is prefixed by \lstinline{#}.
  \item A constant ---~for example an \lstinline{enum} \emph{value}~--- is prefixed by \lstinline{%}.
  \item A function is suffixed by \lstinline{()}.
\end{itemize}

GTK-Doc can parse those special comments and generate HTML pages that can then be easily navigated by an API browser like Devhelp. But the specially-formatted comments in the code are not the only thing that GTK-Doc needs, it also needs integration to the build system of your project (for example the Autotools), alongside some other files to list the different pages, describe the general structure with the list of symbols and optionally provide additional content written in the DocBook XML format. Those files are usually present in the \path{docs/reference/} directory.

Describing in detail how to integrate GTK-Doc support in your code is beyond the scope of this book. For that, you should refer to the GTK-Doc manual~\cite{gtk-doc}.

Every GLib-based library should have a GTK-Doc documentation. But it is also useful to write a GTK-Doc documentation for the internal code of an application. As explained in section~\ref{intro-backend-frontend-separation} p.~\pageref{intro-backend-frontend-separation}, it is a good practice to separate the backend of an application from its frontend(s), and write the backend as an internal library, or, later, a shared library. As such, it is recommended to document the public API of the backend with GTK-Doc, even if it is still an internal, statically-linked library. Because when the codebase becomes bigger, it is a great help -- especially for newcomers -- to have an overview of the available classes, and to know how to use a class without the need to understand its implementation.

\subsection{GObject Introspection Annotations}
The GTK-Doc comment for the \lstinline{myapp_spell_checker_get_suggestions()} function contains GObject Introspection annotations for the return value:
\begin{lstlisting}
/**
 * ...
 * Returns: (transfer full) (element-type utf8): the list of suggestions.
 */
\end{lstlisting}

The function return type is \lstinline{GSList *}, which is not sufficient for language bindings to know what the list contains, and whether and how to free it. \texttt{(transfer full)} means that the return value must be fully freed by the caller: the list itself \emph{and} its elements. \texttt{(element-type utf8)} means that the list contains UTF-8 strings.

For a return value, if the transfer annotation is \texttt{(transfer container)}, the caller needs to free only the data structure, not its elements. And if the transfer annotation is \texttt{(transfer none)}, the ownership of the variable content is not transferred, and thus the caller must not free the return value.

There are many other GObject Introspection (GI) annotations, to name just a couple others:
\begin{itemize}
  \item \texttt{(nullable)}: the value can be \lstinline{NULL};
  \item \texttt{(out)}: an ``out'' function parameter, i.e. a parameter that returns a value.
\end{itemize}

As you can see, the GI annotations are not just useful for language bindings, they also gather useful information to the C programmer.

For a comprehensive list of GI annotations, see the GObject Introspection wiki~\cite{gobject-introspection}.

\subsection{Static vs Non-Static Functions}
In the example code, you can see that the \lstinline{load_dictionary()} function has been marked as \lstinline{static}. It is in fact a good practice in C to mark internal functions as \lstinline{static}. A \lstinline{static} function can be used only in the same *.c file. On the other hand, a public function should be non-static and have a prototype in a header.

There is the \texttt{-Wmissing-prototypes} GCC warning option to ensure that a piece of code follows this convention\footnote{When using the Autotools, the \texttt{AX\_COMPILER\_FLAGS} Autoconf macro enables, among other things, that GCC flag.}.

Also, contrarily to a public function, a \lstinline{static} function doesn't require to be prefixed by the project and class namespaces (here, \lstinline{myapp_spell_checker}).

\subsection{Defensive Programming}
Each public function checks its preconditions with the \lstinline{g_return_if_fail()} or \lstinline{g_return_val_if_fail()} debugging macros, as described in section~\ref{glib-debugging-macros} p.~\pageref{glib-debugging-macros}. The \emph{self} parameter is also checked, to see if it is not \lstinline{NULL}, except for the destructor, \lstinline{myapp_spell_checker_free()}, to be consistent with \lstinline{free()} and \lstinline{g_free()} which accept a \lstinline{NULL} parameter for convenience.

Note that the \lstinline{g_return} macros should be used only for the entry points of a class, that is, in its public functions. You can usually assume that the parameters passed to a \lstinline{static} function are valid, especially the \emph{self} parameter. However, it is sometimes useful to check an argument of a \lstinline{static} function with \lstinline{g_assert()}, to make the code more robust and self-documented.

\subsection{Coding Style}
Finally, it is worth explaining a few things about the coding style. You are encouraged to use early on the same coding style for your project, because it can be something difficult to change afterwards. If every program has different code conventions, it's a nightmare for someone willing to contribute.

Here is an example of a function definition:
\begin{lstlisting}
gboolean
myapp_spell_checker_check_word (MyappSpellChecker *checker,
                                const gchar       *word,
                                gssize             word_length)
{
  /* ... */
}
\end{lstlisting}

See how the parameters are aligned: there is one parameter per line, with the type aligned on the opening parenthesis, and with the names aligned on the same column. Some text editors can be configured to do that automatically.

For a function \emph{call}, if putting all the parameters on the same line results in a too long line, the parameters should also be aligned on the parenthesis, to make the code easier to read:
\begin{lstlisting}
  function_call (one_param,
                 another_param,
                 yet_another_param);
\end{lstlisting}

Unlike other projects like the Linux kernel, there is not really a limit on the line length. A GObject-based code can have quite lengthy lines, even if the parameters of a function are aligned on the parenthesis. Of course if a line ends at, say, column 120, it might mean that there are too many indentation levels and that you should extract the code into intermediate functions.

The return type of a function \emph{definition} is on the previous line than the function name, so that the function name is at the beginning of the line. When writing a function \emph{prototype}, the function name should never be at the beginning of a line, it should ideally be on the same line as the return type\footnote{In \path{myapp-spell-checker.h}, the function names are indented with two spaces instead of being on the same lines as the return types, because there is not enough horizontal space on the page.}. That way, you can do a regular expression search to find the implementation of a function, for example with the \shellcmd{grep} shell command:

\begin{lstlisting}[language=bash]
$ grep -n -E "^function_name" *.c
\end{lstlisting}

Or, if the code is inside a Git repository, it's a little more convenient to use \lstinline[language=bash]{git grep}:

\begin{lstlisting}[language=bash]
$ git grep -n -E "^function_name"
\end{lstlisting}

Likewise, there is an easy way to find the declaration of a public \lstinline{struct}. The convention is to prefix the type name by an underscore when declaring the \lstinline{struct}. For example:
\begin{lstlisting}
/* In the header: */
typedef struct _MyappSpellChecker MyappSpellChecker;

/* In the *.c file: */
struct _MyappSpellChecker
{
  /* ... */
};
\end{lstlisting}

As a result, to find the full declaration of the \lstinline{MyappSpellChecker} type, you can search ``\lstinline{_MyappSpellChecker}'':

\begin{lstlisting}[language=bash]
$ git grep -n _MyappSpellChecker
\end{lstlisting}

In GLib/GTK+, this underscore-prefix convention is normally applied to every \lstinline{struct} that has a separate \lstinline{typedef} line. The convention is not thoroughly followed when a \lstinline{struct} is used in only one *.c file. And the convention is usually not followed for \lstinline{enum} types. However, for your project, nothing prevents you from applying consistently the underscore-prefix convention to all types.

To learn more about the coding style used in GLib, GTK+ and GNOME, read the GNOME Programming Guidelines~\cite{gnome-programming-guidelines}.
