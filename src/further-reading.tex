\chapter{Further Reading}
\label{further-reading}

At this point you should know the bases of GLib core and GObject. You don't need to know \emph{everything} about GLib core and GObject to continue, but having at least a basic understanding will allow you to more easily learn GTK+ and GIO, or any other GObject-based library for that matter.

\section{GTK+ and GIO}
GTK+ and GIO can be learned in parallel.

You should be able to use any GObject class in GIO, just read the class description and skim through the list of functions to have an overview of what features a class provides. Among other interesting things, GIO includes:
\begin{itemize}
  \item \lstinline{GFile} to handle files and directories.
  \item \lstinline{GSettings} to store application settings.
  \item \lstinline{GDBus} -- a high-level API for the D-Bus inter-process communication system.
  \item \lstinline{GSubprocess} for launching child processes and communicate with them asynchronously.
  \item \lstinline{GCancellable}, \lstinline{GAsyncResult} and \lstinline{GTask} for using or implementing asynchronous and cancellable tasks.
  \item Many other features, like I/O streams, network support or application support.
\end{itemize}

For building graphical applications with GTK+, don't panic, the reference documentation has a Getting Started guide, available with Devhelp or online at:\\
\url{https://developer.gnome.org/gtk3/stable/}

After reading the Getting Started guide, skim through the whole API reference to get familiar with the available widgets, containers and base classes. Some widgets have a quite large API, so a few external tutorials are also available, for example for \lstinline{GtkTextView} and \lstinline{GtkTreeView}. See the documentation page on:\\
\url{http://www.gtk.org}

There is also a series of small tutorials on various GLib/GTK+ topics:\\
\url{https://wiki.gnome.org/HowDoI}

\section{Autotools}

A Makefile is generally not sufficient if you want to install your application on different systems. The Autotools (Autoconf, Automake and Libtool) is what GNOME modules use. There are some macros available for e.g. the user documentation, code coverage statistics for unit tests, etc. The most recent book on the subject is \emph{Autotools}, by John~Calcote \cite{autotools}.

\section{Programming Best-Practices}

It is recommended to follow the GNOME Programming Guidelines~\cite{gnome-programming-guidelines}.

The following list is a bit unrelated to GLib/GTK+ development, but is useful for any programming project. After having some practice, it is interesting to learn more about programming \emph{best}-practices. Writing code of good quality is important for preventing bugs and for maintaining a software in the long-run.

\begin{itemize}
  \item \emph{The} book on programming best-practices is \emph{Code Complete}, by Steve~McConnell \cite{code-complete}. Highly recommended\footnote{Although the editor of \emph{Code Complete} is Microsoft Press, the book is not related to Microsoft or Windows. The author explains sometimes stuff related to open source, UNIX and Linux, but one can regret the total absence of the mention ``free/libre software'' and all the benefits of freedom, in particular for this kind of book: being able to learn by reading other's code. But if you are here, you hopefully already know all of this.}.

  \item For guidelines about OOP specifically, see \emph{Object-Oriented Design Heuristics}, by Arthur~Riel \cite{oop-book}.

  \item An excellent source of information is the website of Martin~Fowler: refactoring, agile methodology, code design, ...\\
  \url{http://martinfowler.com/}
\end{itemize}

More related to GNOME, Havoc~Pennington's articles have good advises worth the reading, including ``\emph{Working on Free Software}'', ``\emph{Free software UI}'' and ``\emph{Free Software Maintenance: Adding Features}'':\\
\url{http://ometer.com/writing.html}
